%%%%%%%%%%%%%%%%%%%%%%%%%%%%%%%%%%%%%%%%%%%%%%%%%%%%%%%%
% Canonical : https://github.com/lduran2/ece3413_classical_control_systems/doc/lab0405.tex
% Lab report
% By        : Leomar Duran <https://github.com/lduran2>
% When      : 2022-03-28t22:09Q
% For       : ECE 3413
% Version   : 1.1.0
%%%%%%%%%%%%%%%%%%%%%%%%%%%%%%%%%%%%%%%%%%%%%%%%%%%%%%%%
\documentclass[11pt]{article}
\usepackage[utf8]{inputenc}

\usepackage{lib/ccsreport}

\begin{document}

\title{ECE 3413 Lab 04/05 Time Response of First- and\\* Second-Order Systems}
\author{Leomar Durán}
\date{28\(^{\text{th}}\) March 2022}

\maketitle

\section*{Revision History}

\begin{tabularx}\linewidth{@{}rlrX@{}}
    \toprule
        Revision \#
            & Author
            & Revision date
            & Comments
    \\*
    \midrule
        1.1.0
            & Leomar Durán
            & 2022-03-28t22:09Q
            & introduction
    \\*
        1.0.0
            & Leomar Durán
            & 2022-03-28t21:39Q
            & initial lab 04/05
    \\*
        0.0.0
            & Leomar Durán
            & 2022-01-31t00:00R
            & template complete
    \\*
    \bottomrule
\end{tabularx}

\section{Introduction}

The purpose of this lab is
to evaluate the effects of poles, zeros and the gain
on first-order and second-order control systems.

After this lab, students will be able
to determine the effects of poles, zeros, and gain
as well the imaginary and real parts of poles,
and the damping ratio
on overshoot, settling time, rise time, peak time
and the overall shape of the step response.

Additionally, students will be able to plot the transient responses of systems, which include
the impulse response, step response, ramp response, and parabola response. The MATLAB \mintinline{matlab}{linearSystemAnalyzer} is also introduced as a tool for analysis.

\section{Procedure}

\begin{adjustwidth}{0.5in}{0.5in}
    \subsection{Part I}
    \subsection{Part II}
    \subsection{Part III}
\end{adjustwidth}

\section{Results}
\section{Discussion}

\newpage
\appendix
\title{Appendix}\label{doc:apx}
\maketitle

\section{Codes and Commands used in the lab}

\begin{enumerate}
    \item
        \mintinline{matlab}{conv}
        \tabto{1.5in}
        \(\Rightarrow\) convolve or multiply polynomials
\end{enumerate}

\end{document}
