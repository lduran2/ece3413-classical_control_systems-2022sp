%%%%%%%%%%%%%%%%%%%%%%%%%%%%%%%%%%%%%%%%%%%%%%%%%%%%%%%%
% Canonical : https://github.com/lduran2/ece3413_classical_control_systems/doc/lab0405.tex
% Lab report
% By        : Leomar Duran <https://github.com/lduran2>
% When      : 2022-03-29t20:56Q
% For       : ECE 3413
% Version   : 1.6.0
%%%%%%%%%%%%%%%%%%%%%%%%%%%%%%%%%%%%%%%%%%%%%%%%%%%%%%%%
\documentclass[11pt]{article}
\usepackage[utf8]{inputenc}

\usepackage{lib/ccsreport}

\begin{document}

\title{ECE 3413 Lab 04/05 Time Response of First- and\\* Second-Order Systems}
\author{Leomar Durán}
\date{28\(^{\text{th}}\) March 2022}

\maketitle

\section*{Revision History}

\begin{tabularx}\linewidth{@{}rlrX@{}}
    \toprule
        Revision \#
            & Author
            & Revision date
            & Comments
    \\*
    \midrule
        1.6.0
            & Leomar Durán
            & 2022-03-29t20:56Q
            & summary for real, imaginary
    \\*
        1.5.0
            & Leomar Durán
            & 2022-03-29t18:24Q
            & solved for real, imaginary parts
    \\*
        1.4.0
            & Leomar Durán
            & 2022-03-29t15:20Q
            & second-order peak time, overshoot
    \\*
        1.3.0
            & Leomar Durán
            & 2022-03-29t06:12Q
            & steady-state normalization, start second-order
    \\*
        1.2.0
            & Leomar Durán
            & 2022-03-29t00:05Q
            & explaining model of \(G_1(\cdot;\cdot)\)
    \\*
        1.1.0
            & Leomar Durán
            & 2022-03-28t22:09Q
            & introduction
    \\*
        1.0.0
            & Leomar Durán
            & 2022-03-28t21:39Q
            & initial lab 04/05
    \\*
        0.0.0
            & Leomar Durán
            & 2022-01-31t00:00R
            & template complete
    \\*
    \bottomrule
\end{tabularx}

\section{Introduction}

The purpose of this lab is
to evaluate the effects of poles, zeros and the gain
on first-order and second-order control systems.

After this lab, students will be able
to determine the effects of poles, zeros, and gain
as well the imaginary and real parts of poles,
and the damping ratio
on overshoot, settling time, rise time, peak time
and the overall shape of the step response.

Additionally, students will be able to plot the transient responses of systems, which include
the impulse response, step response, ramp response, and parabola response. The MATLAB \mintinline{matlab}{linearSystemAnalyzer} is also introduced as a tool for analysis.

\section{Procedure}

\begin{adjustwidth}{0.5in}{0in}
    \subsection{Part I}
        \subsubsection{Poles in first-order systems}
        In Part I, we will practice by first performing modifications to the parametric first-order system
        \[
            G_1\brao{s; a} := \frac{a}{s + a},
        \]
        and analyze the resulting effects with the \mintinline{matlab}{linearSystemAnalyzer}.

        We will analyze \(G_1\brao{s;a}\) for \(a \in \brac{1..4}\). These systems are modelled by the Simulink subsystem in Fig. \ref{fig:1st order pole}.

        \begin{figure}
            \centering
            \includegraphics[%
                page=2,%
                trim=3.375in 2.125in 3.375in 2.0625in,%
                clip,%
                width=0.75\linewidth%
            ]{lab0405/fig/time_response_slx1.pdf}%
            \caption{Part I: Manipulating the pole in a First-Order System.}
            \label{fig:1st order pole}
        \end{figure}

        Effectively, this is manipulation of the pole of \(G_1\brao{s;a}\). The gain is also changed simultaneously to keep the steady state value at \(1\), thus normalizing with the steady state value. The steady state value is specified by \(\lim_{s\to0} G(s)\).

        Without this normalization, we would have \(G\brao{s;a} = \frac1{s + a}\), and its stead state value
        \[
            G(s) \xrightarrow{s\to0} \frac1{\brao0 + a} = \frac1a.
        \]
        Thus we divide by \(\lim_{s\to0} G(s)\) or \(\frac1a\) to normalize
        \[
            G_1\brao{s;a} = \frac{1}{s + a} \div \frac1a = \frac{a}{s + a}.
        \]

        \subsubsection{Poles in second-order systems}

        Next in Part I, we analyze the manipulation of the poles in the second-order function
        \[
            G_2\brao{s; a,b} := \frac{b}{s^2 + as + b}
        \]
        Specifically, we will change independently the real \(x\) and imaginary \(y\) parts of the pole \(x + y\mathcal{j}\), and the natural frequency \(\omega_n\), which relates to the pole \(2\zeta\omega_n\).

        \begin{figure}
            \centering
            \includegraphics[%
                page=3,%
                trim=2.75in 2.6875in 2.75in 2.625in,%
                clip,%
                width=0.75\linewidth%
            ]{lab0405/fig/time_response_slx1.pdf}%
            \caption{Caption}
            \label{fig:2nd order pole}
        \end{figure}

        The analysis for the second-order systems in Part I, all use models with the form described in Fig. \ref{fig:2nd order pole}. However, with the values in \(G_b\) and \(G_a\) changed depending on the analysis.

        In exercises 2 and 3 respectively, we vary the real part of the pole and the imaginary part of the pole, respectively, each by multiplying it by a coefficient.

        Well, the poles of \(G_2\brao{s; a,b} = \frac{B}A\brao{s}\) have
        \[
            \begin{aligned}
                \brao*{
                    \mathfrak{Re}\Brac{s} \middle| A(s) = 0%
                    \,
                } &{}= \brao*{\frac{-a}{2}}\brac*{
                    \begin{matrix}
                        1 \\* 1 \\*
                    \end{matrix}
                },
            \\*
                \brao*{
                    \mathfrak{Im}\Brac{s} \middle| A(s) = 0%
                    \,
                } &{}= \brao*{\sqrt{b - \brao*{\frac{a}2}^2}}\brac*{
                    \begin{matrix}
                        1 \\* -1 \\*
                    \end{matrix}
                }.
            \\*
            \end{aligned}
        \]

        Now let \(
            k, l \in \mathbb{C},
            \setprime{A} : \mathbb{C} \to \mathbb{C}
        \), s.t.
        \(
            \brao*{\mathfrak{Re}\Brac{s} \middle| \setprime{A}\brao{s} = 0} = k\brao*{\mathfrak{Re}\Brac{s} \middle| A\brao{s} = 0}
        \),
        \(
            \brao*{\mathfrak{Im}\Brac{s} \middle| \setprime{A}\brao{s} = 0} = l\brao*{\mathfrak{Im}\Brac{s} \middle| A\brao{s} = 0}
        \),
        then
        \[
            \setprime{{\vec{s}\,{\vphantom{1^1}}}}
            := \brao*{s \middle| \setprime{A}\brao{s} = 0}
            = \brac*{
                \begin{matrix}
                    1 & \mathcal{j} \\* 1 & -\mathcal{j} \\*
                \end{matrix}
            }\brac*{
                \begin{matrix}
                    k\brao*{\frac{-a}{2}} \\*
                    l\sqrt{b - \brao*{\frac{a}2}^2}
                    \\*
                \end{matrix}
            }.
        \]
        These are the roots of \(
            s^2 + kas + \brao*{l^2 b + \brao*{k^2 - l^2}\brao*{\frac{a}2}^2}
            % = s^2 + 2\zeta\omega_ns + \omega_n^2
        \).
        %So \(2\zeta\omega_n = ka\)
        %and \(\omega_n^2 = \brao{k^2 - l^2}\frac{a^2}4 + l^2 b\).
        % Thus the natural frequency and damping ratio respectively
        % \[
            % \begin{gathered}
                % \omega_n = \sqrt{l^2 b + \brao*{k^2 - l^2}\brao*{\frac{a}2}^2},
            % \\*
                % \zeta = \frac{ka}{2\omega_n}.
            % \\*
            % \end{gathered}
        % \]

        % Let's further multiply
        % \(\setprime[2]{\omega_n} = m\omega_n\).

        % In which case, these are the roots of \(
            % s^2 + kas + m\brao*{\brao*{k^2 - l^2}\frac{a^2}4 + l^2 b}
            % = 2\zeta\omega_ns + \setprime[2]{\omega_n^2}
        % \).
        % \[
            % \begin{gathered}
                % \setprime[2]{\omega_n} = m\sqrt{\brao{k^2 - l^2}\frac{a^2}4 + l^2 b},
            % \\*
                % \zeta = \frac{ka}{2\setprime[2]{\omega_n^2}}.
            % \\*
            % \end{gathered}
        % \]

        % From this, we can see that
        % \begin{enumerate}
            % \item
                % real part through \(k\) is proportional to the natural frequency \(w_n\), but changing it maintains the damping ratio \(\zeta\) at a similar value. Thus
                % \begin{enumerate}
                    % \item
                        % peak time \(T_p\) and settling time \(T_s\) decrease as the real part increases.
                    % \item
                        % there is very little effect on overshoot \(\si\percent OS\) as the real part changes.
                % \end{enumerate}
            % \item
                % it must be that \(b > \brao{l^{-2} - 1}\hfrac{a^2}4\). So the imaginary part through \(l\) is proportional to the natural frequency \(w_n\) and inversely proportional to the  the damping ratio \(\zeta\).
        % \end{enumerate}

        \paragraph{Peak time}
        Then the new peak time for \(s \in \setprime{\mathbf{s}}\),
        \[
            \setprime{T}_p
            := \frac\pi{\abs{\mathfrak{Im}\Brac{s}}}
            = \frac\pi{l\sqrt{b - \brao*{\frac{a}2}^2}}
            = \frac{1}l \brao*{\frac\pi{\sqrt{b - \brao*{\frac{a}2}^2}}}
            = \frac{T_p}l.
        \].

        We see that the coefficient of the real part \(k\) has no effect on the peak time. However, the coefficient of the imaginary part \(l\) is inversely proportional to peak time.

        \paragraph{Overshoot}
        Now the new overshoot
        \[
                \setprime{OS}
                := \exp\brao*{\pi\brao*{\frac{\mathfrak{Re}}{\abs{\mathfrak{Im}}}}\Brac{s}}
                = \exp\brao*{\brao\pi\frac
                    {k\brao*{\frac{-a}2}}
                    {l\sqrt{b - \brao*{\frac{a}2}^2}}
                }
                = \exp\brao*{\frac{k}l\brao*{\brao\pi\frac
                    {\brao*{\frac{-a}2}}
                    {\sqrt{b - \brao*{\frac{a}2}^2}}
                }}
        \]
        is necessary and sufficient for
        \[
            \ln\brao{\setprime{OS}} = \frac{k}l\ln\brao*{OS}.
        \]
        Thus we see that \(\hfrac{k}l\) is proportional to the logarithm of the overshoot.

        \paragraph{Constraint for settling time}
        Finally, the settling time
        \[
            T_s
            := \frac{\ln\brao*{\SI2\percent \sqrt{1 - \zeta^2}}}{\mathfrak{Re}\Brac{s}}.
        \]

        Well, let's assume, \(T_s\) is a normal distribution with the approximation
        \[
            \widehat{T_s}
            := \frac{\ln\brao{\SI2\percent}}{\mathfrak{Re}\Brac{s}}.
        \]
        For all \(\zeta \in \brac*{0,1}\), the rate of error for this approximation
        \[
             \frac{T_s - \widehat{T_s}}{T_s}
             = -\brao*{\frac{2\ln\brao{\SI2\percent}}{\ln\brao*{1 - \zeta^2}} + 1}^{-1}
             < 0.
        \]
        So, the error is in the right tail of the distribution. An outlier in a normal distribution is approximated as a value \(1.5\) standard deviations from the mean. We expect a proportion \(0.9332\) of data is within this range. Thus, we find a \(\max\Brac\zeta\) satisfying
        \[
            \frac{T_s - \widehat{T_s}}{T_s}
            \leq \num{0.9332} - 1
            < 0.
        \]
        This \(\max\Brac\zeta \approx \num{0.654847}\), which gives a settling time with an error within \(\SI{6.68}\percent\).

        Now generally
        \[
            \begin{aligned}
                \omega_n
                    &{}:= \sqrt{\mathfrak{Re}^2\Brac{s} + \abs{\mathfrak{Im}}^2\Brac{s}},
            \\*
                \zeta
                    &{}:= \frac{-\mathfrak{Re}\Brac{s}}{\omega_n}.
            \\*
            \end{aligned}
        \]

        So we have
        \[
            \begin{aligned}
                \setprime{\omega}_n
                    &{}= \sqrt{k^2 \mathfrak{Re}^2\Brac{s} + l^2 \abs{\mathfrak{Im}}^2\Brac{s}},
            \\*
                \setprime\zeta
                    &{}= \frac{-k\mathfrak{Re}\Brac{s}}{\omega_n}
                    = \frac{1}{\sqrt{1 + \brao*{
                        \frac{l}k
                        \brao*{\frac
                            {\abs{\mathfrak{Im}}}
                            {\mathfrak{Re}}
                        }
                        \Brac{s}
                    }^2}}.
            \\*
            \end{aligned}
        \]

        Thus, we choose coefficients \(k, l\) s.t.
        \[
            \frac{l}k
            \geq \frac{\sqrt{\frac1{\zeta^2} - 1}}{\brao*{\mathfrak{\abs{Im}/Re}}\Brac{s}}
            = \frac{\sqrt{\frac1{\brao{\num{0.654847}}^2} - 1}}{\brao*{\mathfrak{\abs{Im}/Re}}\Brac{s}}
            = \frac
                {-1.154104}
                {\brao*{\mathfrak{\abs{Im}/Re}}\Brac{s}}.
        \]

        Given our system \(G_2\brao{s; a, b}\), where we have
        \(a = 4\) and \(b = 25\), then \(
            \mathfrak{Re}\Brac{s} = -4/2 = -2
        \) and \(
            \abs{\mathfrak{Im}}\Brac{s} = \sqrt{25 - \brao{4/2}^2} = \sqrt{21}
        \). Thus \(\frac{l}k \geq \num{0.503692}\).

        For fixed \(l = 1\),
        then \(k \leq {2.0}\),
        whereas for fixed \(k = 1\),
        then \(l \geq {0.50}\).

        We will be be using fixed \(l = 1\)
        and \(k \in \brao{1, 2, 1/2}\),
        each of the latter of which fulfills \(k \leq 2.0\).
        Then we will use fixed \(k = 1\)
        and \(l \in \brao{1, 2, 4}\),
        each of the latter of which fulfills \(l \geq 0.50\). Thus, the constraint of \(\max\Brac\zeta \approx \num{0.654847}\) is met.

        \paragraph{Settling time}
        Thus, we can use the approximation \(\widehat{T_s}\) as a stand-in for \(T_s\). Therefore
        \[
            \widehat{\setprime{T}_s}
            := \frac{\ln\brao{\SI2\percent}}{k\mathfrak{Re}\Brac{s}}
            = \frac1k \brao*{\frac{\ln\brao{\SI2\percent}}{\mathfrak{Re}\Brac{s}}}
            = \frac{\widehat{T_s}}k.
        \]

        Thus the coefficient of the imaginary part of the poles \(l\) has no effect on settling time.
        However, the coefficient of the real part \(k\) is inversely proportional to settling time.

        \paragraph{Summary}
        We have the following equations
        \begin{eqnarray}
            \llap{\(
                \setprime{T}_p
            \)} = \rlap{\(
                \frac{T_p}l,
            \)}
        \\*
            \llap{\(
                \ln\brao{\setprime{OS}}
            \)} = \rlap{\(
                \frac{k}l\ln\brao*{OS},
            \)}
        \\*
            \llap{\(
                \widehat{\setprime{T}_s}
            \)} = \rlap{\(
                \frac{\widehat{T_s}}k.
            \)}
        \end{eqnarray}

        Therefore, the coefficient of the imaginary part \(l\) is inversely proportional to the peak time \(\setprime{T}_p\),
        the ratio \(\frac{k}l\) is proportional to overshoot \(\setprime{OS}\),
        and the real part \(k\) is inversely proportional to the settling time \(\setprime{T}_s\).
    \subsection{Part II}
    \subsection{Part III}
\end{adjustwidth}

\section{Results}
\section{Discussion}

\newpage
\appendix
\title{Appendix}\label{doc:apx}
\maketitle

\section{Codes and Commands used in the lab}

\begin{enumerate}
    \item
        \mintinline{matlab}{conv}
        \tabto{1.5in}
        \(\Rightarrow\) convolve or multiply polynomials
\end{enumerate}

\end{document}
